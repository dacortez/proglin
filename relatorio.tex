% Tipo de documento
\documentclass[12pt,a4paper]{article}

% Pacotes
%\usepackage{showlabels}
\usepackage{latexsym}
\usepackage{amsfonts}
\usepackage{amsmath}
\usepackage{amscd}
\usepackage[brazil]{babel}
\usepackage[utf8]{inputenc}
\usepackage[pdftex]{graphicx}

% Paginação
%\textwidth 15.0cm                    % Largura
%\textheight 22.0cm                   % Altura
%\addtolength{\oddsidemargin}{0.0cm}  % Margem esquerda (impar)
%\addtolength{\evensidemargin}{0.0cm} % Margem esquerda (par)
%\addtolength{\topmargin}{0.0cm}      % Margem superior

% Estilo dos parágrafos
\sloppy                              % Mais flexível
\setlength{\jot}{13pt}               % Distância entre linhas do eqnarray
\setlength{\parskip}{1ex}            % Distância entre parágrafos
\renewcommand{\baselinestretch}{1.0} % Distância entre linhas

% Comandos customizados
\newcommand{\vet}{\mathbf}                                   % vetor
\newcommand{\ie}{\emph{i.e.}}                                % isto é
\newcommand{\prodint}[2]{\left\langle #1 , #2 \right\rangle} % produto interno
\newcommand{\Fullbox}{{\rule{2.0mm}{2.0mm}}}                 % caixa cheia
\newcommand{\EOS}{\hfill\Box\vspace{-0.2cm}}                 % fim de enunciado
\newcommand{\EOP}{\hfill\Fullbox\vspace{0.2cm}}              % fim de prova
\newcommand{\eps}{\varepsilon}                               % epsilon
\newcommand{\defi}{\: := \: }                                % definição
\newcommand{\del}{\partial}                                  % derivada parcial

% Contagem de equações por seção
\renewcommand{\theequation}{\thesection.\arabic{equation}}

% Contagem de figuras por seção
\renewcommand{\thefigure}{\thesection.\arabic{figure}}

% Contagem de tabelas por seção
\renewcommand{\thetable}{\thesection.\arabic{table}}

% Zerar as contagem em cada seção
\newcommand{\zerar}{\setcounter{equation}{0}\setcounter{figure}{0}\setcounter{table}{0}}

% Operadores
\DeclareMathOperator{\sen}{sen}
\DeclareMathOperator{\tg}{tg}
\DeclareMathOperator{\cotg}{cotg}
\DeclareMathOperator{\im}{im}
\DeclareMathOperator{\arctg}{arctg}
\DeclareMathOperator{\diag}{diag}
\DeclareMathOperator{\sgn}{sgn}
\DeclareMathOperator{\tr}{tr}

% Conjunto de números
\newcommand{\Z}{\mathbb{Z}}
\newcommand{\C}{\mathbb{C}}
\newcommand{\N}{\mathbb{N}}
\newcommand{\Q}{\mathbb{Q}}
\newcommand{\R}{\mathbb{R}}

%%%%%%%%%%%%%%%%%%%%%%%%%%%%%%%%%%%%%%%%%%%%%%%%%%%%%%%%%%%%%%%%%%%%%%%%%%%%%%%%%%%%%%%%%%%%%%%%%%%%

\begin{document}

\title{Programação Linear \\ Relatório EP3 -- Método Simplex}
\author{Daniel Augusto Cortez -- 2960291 \\ Lucas Rodrigues Colucci -- 6920251}
\date{\today}

\maketitle

%%%%%%%%%%%%%%%%%%%%%%%%%%%%%%%%%%%%%%%%%%%%%%%%%%%%%%%%%%%%%%%%%%%%%%%%%%%%%%%%%%%%%%%%%%%%%%%%%%%%

\zerar
\section{Introdução}
\label{sec:introducao}

Apresentamos neste relatório a implementação do método simplex utilizando o tableau, através da 
linguagem de programação Octave. O algoritmo é desenvolvido em detalhes em~\cite{bertsimas}, 
resultando nos seguintes passos:

\begin{verbatim}
\end{verbatim}

%%%%%%%%%%%%%%%%%%%%%%%%%%%%%%%%%%%%%%%%%%%%%%%%%%%%%%%%%%%%%%%%%%%%%%%%%%%%%%%%%%%%%%%%%%%%%%%%%%%%

\zerar
\section{Funções Implementadas}
\label{sec:funcoes}

%%%%%%%%%%%%%%%%%%%%%%%%%%%%%%%%%%%%%%%%%%%%%%%%%%%%%%%%%%%%%%%%%%%%%%%%%%%%%%%%%%%%%%%%%%%%%%%%%%%%

\zerar
\section{Alguns Exemplos}
\label{sec:exemplos}

%%%%%%%%%%%%%%%%%%%%%%%%%%%%%%%%%%%%%%%%%%%%%%%%%%%%%%%%%%%%%%%%%%%%%%%%%%%%%%%%%%%%%%%%%%%%%%%%%%%%

\zerar
\section{Utilização do Programa}
\label{sec:programa}

%%%%%%%%%%%%%%%%%%%%%%%%%%%%%%%%%%%%%%%%%%%%%%%%%%%%%%%%%%%%%%%%%%%%%%%%%%%%%%%%%%%%%%%%%%%%%%%%%%%%

\begin{thebibliography}{99}
	\bibitem{bertsimas} D.~Bertsimas \& J.~N.~Tsitsiklis, {\it Introduction to Linear Optimization}, 
	1997, Athena Scientific.
\end{thebibliography}

%%%%%%%%%%%%%%%%%%%%%%%%%%%%%%%%%%%%%%%%%%%%%%%%%%%%%%%%%%%%%%%%%%%%%%%%%%%%%%%%%%%%%%%%%%%%%%%%%%%%

\end{document}

