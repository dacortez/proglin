% Tipo de documento
\documentclass[12pt,a4paper]{article}

% Pacotes
%\usepackage{showlabels}
\usepackage{latexsym}
\usepackage{amsfonts}
\usepackage{amsmath}
\usepackage{amscd}
\usepackage[brazil]{babel}
\usepackage[utf8]{inputenc}
\usepackage[pdftex]{graphicx}

% Paginação
%\textwidth 18.0cm                    % Largura
%\textheight 23.0cm                   % Altura
%\addtolength{\oddsidemargin}{-2.3cm}  % Margem esquerda (impar)
%\addtolength{\evensidemargin}{-2.3cm} % Margem esquerda (par)
%\addtolength{\topmargin}{-1.0cm}      % Margem superior

% Estilo dos parágrafos
\sloppy                              % Mais flexível
\setlength{\jot}{13pt}               % Distância entre linhas do eqnarray
\setlength{\parskip}{1ex}            % Distância entre parágrafos
\renewcommand{\baselinestretch}{1.0} % Distância entre linhas

% Comandos customizados
\newcommand{\vet}{\mathbf}                                   % vetor
\newcommand{\ie}{\emph{i.e.}}                                % isto é
\newcommand{\prodint}[2]{\left\langle #1 , #2 \right\rangle} % produto interno
\newcommand{\Fullbox}{{\rule{2.0mm}{2.0mm}}}                 % caixa cheia
\newcommand{\EOS}{\hfill\Box\vspace{-0.2cm}}                 % fim de enunciado
\newcommand{\EOP}{\hfill\Fullbox\vspace{0.2cm}}              % fim de prova
\newcommand{\eps}{\varepsilon}                               % epsilon
\newcommand{\defi}{\: := \: }                                % definição
\newcommand{\del}{\partial}                                  % derivada parcial

% Contagem de equações por seção
\renewcommand{\theequation}{\thesection.\arabic{equation}}

% Contagem de figuras por seção
\renewcommand{\thefigure}{\thesection.\arabic{figure}}

% Contagem de tabelas por seção
\renewcommand{\thetable}{\thesection.\arabic{table}}

% Zerar as contagem em cada seção
\newcommand{\zerar}{\setcounter{equation}{0}\setcounter{figure}{0}\setcounter{table}{0}}

% Operadores
\DeclareMathOperator{\sen}{sen}
\DeclareMathOperator{\tg}{tg}
\DeclareMathOperator{\cotg}{cotg}
\DeclareMathOperator{\im}{im}
\DeclareMathOperator{\arctg}{arctg}
\DeclareMathOperator{\diag}{diag}
\DeclareMathOperator{\sgn}{sgn}
\DeclareMathOperator{\tr}{tr}

% Conjunto de números
\newcommand{\Z}{\mathbb{Z}}
\newcommand{\C}{\mathbb{C}}
\newcommand{\N}{\mathbb{N}}
\newcommand{\Q}{\mathbb{Q}}
\newcommand{\R}{\mathbb{R}}

%%%%%%%%%%%%%%%%%%%%%%%%%%%%%%%%%%%%%%%%%%%%%%%%%%%%%%%%%%%%%%%%%%%%%%%%%%%%%%%%%%%%%%%%%%%%%%%%%%%%

\begin{document}

\title{Programação Linear \\ Relatório EP3 -- Método Simplex}
\author{Daniel Augusto Cortez -- 2960291 \\ Lucas Rodrigues Colucci -- 6920251}
\date{\today}

\maketitle

%%%%%%%%%%%%%%%%%%%%%%%%%%%%%%%%%%%%%%%%%%%%%%%%%%%%%%%%%%%%%%%%%%%%%%%%%%%%%%%%%%%%%%%%%%%%%%%%%%%%

\zerar
\section{Introdução}
\label{sec:introducao}

Apresentamos neste relatório a implementação do método simplex (full tableau), utilizando a 
linguagem de programação Octave. O algoritmo é desenvolvido em detalhes em~\cite{bertsimas}.
Considera-se o problema de programação linear em seu formato padrão
%
\begin{equation} \label{eq:PL}
	\begin{array}{rl}
		\text{minimizar} & \vet{c}^T \vet{x} \\
		\text{sujeito a} & \vet{A} \vet{x} = \vet{b} \\\
		                 & \vet{x} \geq \vet{0} \, ,
	 \end{array}
\end{equation}
%
onde $\vet{x}, \vet{c} \in \R^n$, $\vet{b} \in \R^m$ e $\vet{A} \in \R^{m \times n}$. 
O algoritmo pode ser descrito pelos seguintes passos:

\vspace{0.5cm}

\noindent {\bf Fase I.}
\vspace{-0.2cm}
\begin{enumerate}
	\item Multiplique algumas das restrições por $-1$, modificando o problema de tal forma que  
	$\vet{b} \geq 0$.
	\item Introduza variáveis artificiais $y_1, \ldots, y_m$ e aplique o método simplex ao problema
	auxiliar com custo $\sum_{i=1}^{m}y_i$.
	\item Se o custo ótimo do problema auxiliar é positivo, o problema original é inviável e o 
	algoritmo termina.
	\item Se o custo ótimo do problema auxiliar é nulo, uma solução viável do problema original é 
	obtida. Se nenhuma variável artificial está na base final, as variáveis artificiais e respectivas
	colunas são eliminadas, e uma base viável para o problema original está disponível.
	\item Se a $\ell$-ésima variável básica é artificial, examine a $\ell$-ésima entrada das
	colunas $\vet{B}^{-1}\vet{A}_j$, $j = 1, \ldots, n$. Se todas as entradas são nulas, então a 
	$\ell$-ésima linha representa uma restrição redundante e é eliminada. De outra forma, se a 
	$\ell$-ésima entrada da $j$-ésima coluna é diferente de zero, aplique uma mudança de base:
	a $\ell$-ésima variável básica sai e $x_j$ entra na base. Repita esse procedimento até que todas 
	as variáveis artificiais sejam levadas para fora da base.
\end{enumerate}

\noindent {\bf Fase II.}
\vspace{-0.2cm}
\begin{enumerate}
	\item Faça a base e o tableau final obtidos na fase I serem a base e o tableau inicial para a 
	fase~II.
	\item Calcule o custo reduzido de todas as variáveis para essa base inicial, usando o vetor de 
	custos do problema original.
	\item Aplique o método simplex ao problema original.
\end{enumerate}

\noindent {\bf Uma iteração da implementação através do tableau.}
\label{iteracao}
\vspace{-0.2cm}
\begin{enumerate}
	\item Uma iteração típica começa com um tableau associado com uma matriz básica $\vet{B}$ e a 
	solução básica correspondente $\vet{x}$.
	\item Examine todos os custos reduzidos na linha zero do tableau. Se todos forem não-negativos,
	a solução viável básica atual é ótima, e o algoritmo termina. Caso contrário, escolha algum $j$
	tal que $\overline{c}_j < 0$.
	\item Considere o vetor $\vet{u} = \vet{B}^{-1}\vet{A}_j$, que é a $j$-ésima coluna (a coluna 
	pivô) do tableau. Se nenhuma componente de $\vet{u}$ é positiva, o custo ótimo é $-\infty$ e o 
	algoritmo termina.
	\item Para cada $i$ tal que $u_i > 0$, calcule a razão $x_{B(i)}/u_i$. Seja $\ell$ o índice da 
	linha que corresponde à menor razão. A coluna $\vet{A}_{B(\ell)}$ saí da base e a coluna 
	$\vet{A}_j$ entra.
	\item Adicione a cada linha do tableau um múltiplo constante da $\ell$-ésima linha tal que o 
	$u_\ell$ (elemento pivô) se torne um e todas as outras entradas da coluna pivô se tornem zero. 
\end{enumerate}

A implementação do método simplex em Octave seguiu extamente os passos acima, conforme descrevemos
a seguir.

%%%%%%%%%%%%%%%%%%%%%%%%%%%%%%%%%%%%%%%%%%%%%%%%%%%%%%%%%%%%%%%%%%%%%%%%%%%%%%%%%%%%%%%%%%%%%%%%%%%%

\zerar
\section{Funções Implementadas}
\label{sec:funcoes}

No arquivo \verb|simplex.m|, escrevemos a função 
%
\begin{verbatim}
	[ind x] = simplex(A, b, c, m, n, print)
\end{verbatim}  
%
que pode ser chamada externamente para resolver qualquer PL na forma (\ref{eq:PL}). 
Os parâmetros de entrada da função são:
%
\begin{itemize}
	\item $\vet{A}$: matriz de dimensão $m \times n$ de coeficientes das restrições;
	\item $\vet{b}$: vetor de dimensão $m$ do lado direito das restrições;
	\item $\vet{c}$: vetor de dimensão $n$ de custos;
	\item $m$: número de restrições;
	\item $n$: número de variáveis;
	\item \verb|print|: booleano indicando se o tableau deve ser impresso a cada iteração.  
\end{itemize} 
%
Os parâmetros de saída são:
%
\begin{itemize}
	\item \verb|ind|: valor $0$ se o problema tiver solução ótima finita, $-1$ se tiver solução
	ilimitada e $+1$ se o problema for inviável;
	\item $\vet{x}$: solução ótima do problema, quando existir.
\end{itemize}
%
A função \verb|simplex| começa resolvendo a fase I do método simplex. Em seguida, testa se o custo
encontrado na fase I é positivo. Nesse caso, o problema é inviável. Caso contrário, prossegue com
a fase II. 


A fase I do método é resolvida pela função
%
\begin{verbatim}
	[T B m] = phase_1(A, B, m, n, print)
\end{verbatim}
%
Os parâmetros de entrada da função são:
%
\begin{itemize}
	\item $\vet{A}$: matriz de dimensão $m \times n$ de coeficientes das restrições;
	\item $\vet{b}$: vetor de dimensão $m$ do lado direito das restrições;
	\item $m$: número de restrições;
	\item $n$: número de variáveis;
	\item \verb|print|: booleano indicando se o tableau deve ser impresso a cada iteração.  
\end{itemize} 
%
Os parâmetros de saída são:
%
\begin{itemize}
	\item $T$: matriz de dimensão $(m+1) \times (n+1)$ que representa o tableau final da fase 1;
	\item $B$: vetor de índices das variáveis básicas;
	\item $m$: número de linhas do tableau final, excluindo a primeira linha. 
\end{itemize}
%
A função \verb|phase_1| começa colocando o problema (\ref{eq:PL}) no formato $\vet{b} \geq 0$,
depois adiciona as $m$ variáveis artificiais à matriz $\vet{A}$ (colocando uma matriz identidade
$m \times m$ depois da última coluna de $\vet{A}$). Em seguida, monta o tableau inicial em $T$
(implementado na função \verb|phase_1_tableau|). O vetor de índices das variáveis básicas $B$ é
inicialmente definido com os índices das variáveis artificiais. O tableau T é então iterado, 
pivotando de acordo com o método simplex, através da função \verb|tableau_solve|. O tableau
resultante das iterações é testado para verificar a viabilidade do problema original (custo nulo). 
Se o problema for viável, prossegue removendo as colunas das variáveis artificias do tableau, depois
removendo possíveis restrições redundantes (função \verb|remove_redundants|), e finalmente remove
as variáveis artificiais da solução básica encontrada, se estiverem presentes (função 
\verb|exit_artificials|). Essas duas últimas funções implementam o passo 5 da fase I do método 
simplex.

A fase II do método é resolvida pela função
%
\begin{verbatim}
	[ind x] = phase_2(T, B, c, m, n, print)
\end{verbatim}
%
Os parâmetros de entrada da função são:
%
\begin{itemize}
	\item $T$: matriz de dimensão $(m+1) \times (n+1)$ que representa o tableau final da fase 1;
	\item $B$: vetor de índices das variáveis básicas;
	\item $\vet{c}$: vetor de dimensão $n$ de custos;
	\item $m$: número de restrições;
	\item $n$: número de variáveis;
	\item \verb|print|: booleano indicando se o tableau deve ser impresso a cada iteração.  
\end{itemize} 
%
Os parâmetros de saída são:
%
\begin{itemize}
	\item \verb|ind|: valor $0$ se o problema tiver solução ótima finita, $-1$ se tiver solução 
	ilimitada e $+1$ se o problema for inviável;
	\item $\vet{x}$: solução ótima do problema, quanado existir.
\end{itemize}
%
A fase II começa calculando o custo e os custos reduzidos do problema original em função do 
tableau final retornado pela fase I. Esses custos são utilizados para atualizar a primeira linha
do tableau, que é iterado em seguida pela função \verb|tableau_solve|.


Tanto a fase I quanto a fase II utilizam essencialmente a função
%
\begin{verbatim}
	[T B iter ind x] = tableau_solve(T, B, m, n, print, iter)
\end{verbatim}
%
cujos parâmetros de entrada são
%
\begin{itemize}
	\item $T$: matriz de dimensão $(m+1) \times (n+1)$ que representa o tableau da iteração 
	\verb|iter|$ - 1$;
	\item $B$: vetor de índices das variáveis básicas da iteração \verb|iter|$ - 1$;
	\item $m$: número de variáveis básicas;
	\item $n$: número de variáveis;
	\item \verb|print|: booleano indicando se o tableau deve ser impresso a cada iteração.  
	\item \verb|iter|: número da iteração atual.
\end{itemize} 
%
Os parâmetros de saída são:
%
\begin{itemize}
	\item $T$: matriz de dimensão $(m+1) \times (n+1)$ que representa o tableau da iteração 
	\verb|iter|;
	\item $B$: vetor de índices das variáveis básicas da iteração \verb|iter|;
	\item \verb|iter|: número da iteração atual.
	\item \verb|ind|: valor $0$ se o problema tiver solução ótima finita, $-1$ se tiver solução
	ilimitada e $+1$ se o problema for inviável;
	\item $\vet{x}$: solução ótima do problema, quando existir.
\end{itemize}
%
A função acima implementa os passos descritos no algoritmo ``Uma iteração da implementação através 
do tableau" (página~\pageref{iteracao}). A mesma utiliza a regra Bland~\cite{bertsimas} na 
determinação da variável básica a sair da base e da não-básica a entrar na mesma. Com isso,
evita-se a possibilidade do algoritmo ciclar. A operação básica em \verb|tableau_solve| é a 
pivotação do tableau, que é feita pela função \verb|pivot|.


A impressão do tableau durante a execução do programa é feita pela função
%
\begin{verbatim}
	print_tableau(T, B, m, n, i_pivot, j_pivot, iter, msg)
\end{verbatim}
%
Os parâmetros de entrada dessa função são
%
\begin{itemize}
	\item $T$: matriz de dimensão $(m+1) \times (n+1)$ que representa o tableau da iteração 
	\verb|iter|;
	\item $B$: vetor de índices das variáveis básicas da iteração \verb|iter|;
	\item $m$: número de variáveis básicas;
	\item $n$: número de variáveis;
	\item \verb|i_pivot|: linha de pivotamento do tableau (zero caso não aplicável);
	\item \verb|j_pivot|: coluna de pivotamento do tableau (zero caso não aplicável)
	\item \verb|iter|: número da iteração atual.
	\item \verb|msg|: observação a ser impressa ao lado do número da iteração.
\end{itemize} 
%
A função basicamente itera sobre todas as entradas da matriz $T$, imprimindo o seu conteúdo com
a formatação adequada. Caso o elemento iterado corresponda à posição (\verb|i_pivot|, 
\verb|j_pivot|), um símbolo $\ast$ é impresso para marcar o pivô.


As demais funções implementadas são utilizadas para modularizar o código e são de propósito 
específico:
%
\begin{verbatim}
[A b] = make_b_positive(A, b, m, n)
[T B m] = remove_redundants(T, B, m, n, print, iter)
[T B] = exit_artificials(T, B, m, n, print, iter)
bool = is_zero_vector(v)
x = get_solution(T, B, m, n)
T = pivot(T, i_pivot, j_pivot, m, n)
\end{verbatim}
%
Não entraremos em detalhes sobre a implementação das mesmas.

%%%%%%%%%%%%%%%%%%%%%%%%%%%%%%%%%%%%%%%%%%%%%%%%%%%%%%%%%%%%%%%%%%%%%%%%%%%%%%%%%%%%%%%%%%%%%%%%%%%%

\zerar
\section{Alguns Exemplos}
\label{sec:exemplos}

Os exemplos abaixo mostram a saída gerada ao se rodar o script \verb|ep3.m| com o Octave.
(\verb|$ ocatve -qf ep3.m|). O script apresenta a formulação de 7 problemas no formato 
(\ref{eq:PL}) que são resolvidos com a função \verb|simplex| implementada em \verb|simplex.m| e
descrita anteriormente. Os exemplos mostram problemas com uma solução ótima finita (1, 2 e 3),
problemas inviáveis (4 e 5) e problemas ilimitados com custo $-\infty$ (6 e 7).

\begin{tiny}
\begin{verbatim}
-------------------------------------
Exemplo 1 - Solução Ótima - Linhas LI
-------------------------------------
m =  3
n =  6
A =

   1   2   2   1   0   0
   2   1   2   0   1   0
   2   2   1   0   0   1

b =

   20
   20
   20

c =

  -10
  -12
  -12
    0
    0
    0

Simplex: Fase 1

Iteração 1
            | x1      | x2      | x3      | x4      | x5      | x6      | x7      | x8      | x9      |
    -60.000 |  -5.000 |  -5.000 |  -5.000 |  -1.000 |  -1.000 |  -1.000 |   0.000 |   0.000 |   0.000 |
-------------------------------------------------------------------------------------------------------
x7   20.000 |   1.000 |   2.000 |   2.000 |   1.000 |   0.000 |   0.000 |   1.000 |   0.000 |   0.000 |
x8   20.000 |   2.000*|   1.000 |   2.000 |   0.000 |   1.000 |   0.000 |   0.000 |   1.000 |   0.000 |
x9   20.000 |   2.000 |   2.000 |   1.000 |   0.000 |   0.000 |   1.000 |   0.000 |   0.000 |   1.000 |

Iteração 2
            | x1      | x2      | x3      | x4      | x5      | x6      | x7      | x8      | x9      |
    -10.000 |   0.000 |  -2.500 |   0.000 |  -1.000 |   1.500 |  -1.000 |   0.000 |   2.500 |   0.000 |
-------------------------------------------------------------------------------------------------------
x7   10.000 |   0.000 |   1.500 |   1.000 |   1.000 |  -0.500 |   0.000 |   1.000 |  -0.500 |   0.000 |
x1   10.000 |   1.000 |   0.500 |   1.000 |   0.000 |   0.500 |   0.000 |   0.000 |   0.500 |   0.000 |
x9    0.000 |   0.000 |   1.000*|  -1.000 |   0.000 |  -1.000 |   1.000 |   0.000 |  -1.000 |   1.000 |

Iteração 3
            | x1      | x2      | x3      | x4      | x5      | x6      | x7      | x8      | x9      |
    -10.000 |   0.000 |   0.000 |  -2.500 |  -1.000 |  -1.000 |   1.500 |   0.000 |   0.000 |   2.500 |
-------------------------------------------------------------------------------------------------------
x7   10.000 |   0.000 |   0.000 |   2.500*|   1.000 |   1.000 |  -1.500 |   1.000 |   1.000 |  -1.500 |
x1   10.000 |   1.000 |   0.000 |   1.500 |   0.000 |   1.000 |  -0.500 |   0.000 |   1.000 |  -0.500 |
x2    0.000 |   0.000 |   1.000 |  -1.000 |   0.000 |  -1.000 |   1.000 |   0.000 |  -1.000 |   1.000 |

Iteração 4
            | x1      | x2      | x3      | x4      | x5      | x6      | x7      | x8      | x9      |
      0.000 |   0.000 |   0.000 |   0.000 |   0.000 |   0.000 |   0.000 |   1.000 |   1.000 |   1.000 |
-------------------------------------------------------------------------------------------------------
x3    4.000 |   0.000 |   0.000 |   1.000 |   0.400 |   0.400 |  -0.600 |   0.400 |   0.400 |  -0.600 |
x1    4.000 |   1.000 |   0.000 |   0.000 |  -0.600 |   0.400 |   0.400 |  -0.600 |   0.400 |   0.400 |
x2    4.000 |   0.000 |   1.000 |   0.000 |   0.400 |  -0.600 |   0.400 |   0.400 |  -0.600 |   0.400 |

Iteração 5 (tableau final da fase 1)
            | x1      | x2      | x3      | x4      | x5      | x6      |
      0.000 |   0.000 |   0.000 |   0.000 |   0.000 |   0.000 |   0.000 |
-------------------------------------------------------------------------
x3    4.000 |   0.000 |   0.000 |   1.000 |   0.400 |   0.400 |  -0.600 |
x1    4.000 |   1.000 |   0.000 |   0.000 |  -0.600 |   0.400 |   0.400 |
x2    4.000 |   0.000 |   1.000 |   0.000 |   0.400 |  -0.600 |   0.400 |

Simplex: Fase 2

Iteração 1
            | x1      | x2      | x3      | x4      | x5      | x6      |
    136.000 |   0.000 |   0.000 |   0.000 |   3.600 |   1.600 |   1.600 |
-------------------------------------------------------------------------
x3    4.000 |   0.000 |   0.000 |   1.000 |   0.400 |   0.400 |  -0.600 |
x1    4.000 |   1.000 |   0.000 |   0.000 |  -0.600 |   0.400 |   0.400 |
x2    4.000 |   0.000 |   1.000 |   0.000 |   0.400 |  -0.600 |   0.400 |

Solução ótima encontrada com custo -136.00000:
x =

   4
   4
   4
   0
   0
   0


-------------------------------------
Exemplo 2 - Solução Ótima - Linhas LD
-------------------------------------
m =  4
n =  4
A =

   1   2   3   0
  -1   2   6   0
   0   4   9   0
   0   0   3   1

b =

   3
   2
   5
   1

c =

   1
   1
   1
   0

Simplex: Fase 1

Iteração 1
            | x1      | x2      | x3      | x4      | x5      | x6      | x7      | x8      |
    -11.000 |  -0.000 |  -8.000 | -21.000 |  -1.000 |   0.000 |   0.000 |   0.000 |   0.000 |
---------------------------------------------------------------------------------------------
x5    3.000 |   1.000 |   2.000 |   3.000 |   0.000 |   1.000 |   0.000 |   0.000 |   0.000 |
x6    2.000 |  -1.000 |   2.000*|   6.000 |   0.000 |   0.000 |   1.000 |   0.000 |   0.000 |
x7    5.000 |   0.000 |   4.000 |   9.000 |   0.000 |   0.000 |   0.000 |   1.000 |   0.000 |
x8    1.000 |   0.000 |   0.000 |   3.000 |   1.000 |   0.000 |   0.000 |   0.000 |   1.000 |

Iteração 2
            | x1      | x2      | x3      | x4      | x5      | x6      | x7      | x8      |
     -3.000 |  -4.000 |   0.000 |   3.000 |  -1.000 |   0.000 |   4.000 |   0.000 |   0.000 |
---------------------------------------------------------------------------------------------
x5    1.000 |   2.000*|   0.000 |  -3.000 |   0.000 |   1.000 |  -1.000 |   0.000 |   0.000 |
x2    1.000 |  -0.500 |   1.000 |   3.000 |   0.000 |   0.000 |   0.500 |   0.000 |   0.000 |
x7    1.000 |   2.000 |   0.000 |  -3.000 |   0.000 |   0.000 |  -2.000 |   1.000 |   0.000 |
x8    1.000 |   0.000 |   0.000 |   3.000 |   1.000 |   0.000 |   0.000 |   0.000 |   1.000 |

Iteração 3
            | x1      | x2      | x3      | x4      | x5      | x6      | x7      | x8      |
     -1.000 |   0.000 |   0.000 |  -3.000 |  -1.000 |   2.000 |   2.000 |   0.000 |   0.000 |
---------------------------------------------------------------------------------------------
x1    0.500 |   1.000 |   0.000 |  -1.500 |   0.000 |   0.500 |  -0.500 |   0.000 |   0.000 |
x2    1.250 |   0.000 |   1.000 |   2.250 |   0.000 |   0.250 |   0.250 |   0.000 |   0.000 |
x7    0.000 |   0.000 |   0.000 |   0.000 |   0.000 |  -1.000 |  -1.000 |   1.000 |   0.000 |
x8    1.000 |   0.000 |   0.000 |   3.000*|   1.000 |   0.000 |   0.000 |   0.000 |   1.000 |

Iteração 4
            | x1      | x2      | x3      | x4      | x5      | x6      | x7      | x8      |
      0.000 |   0.000 |   0.000 |   0.000 |   0.000 |   2.000 |   2.000 |   0.000 |   1.000 |
---------------------------------------------------------------------------------------------
x1    1.000 |   1.000 |   0.000 |   0.000 |   0.500 |   0.500 |  -0.500 |   0.000 |   0.500 |
x2    0.500 |   0.000 |   1.000 |   0.000 |  -0.750 |   0.250 |   0.250 |   0.000 |  -0.750 |
x7    0.000 |   0.000 |   0.000 |   0.000 |   0.000 |  -1.000 |  -1.000 |   1.000 |   0.000 |
x3    0.333 |   0.000 |   0.000 |   1.000 |   0.333 |   0.000 |   0.000 |   0.000 |   0.333 |

Iteração 5 (removendo variáveis redundantes)
            | x1      | x2      | x3      | x4      |
      0.000 |   0.000 |   0.000 |   0.000 |   0.000 |
-----------------------------------------------------
x1    1.000 |   1.000 |   0.000 |   0.000 |   0.500 |
x2    0.500 |   0.000 |   1.000 |   0.000 |  -0.750 |
x3    0.333 |   0.000 |   0.000 |   1.000 |   0.333 |

Iteração 5 (tableau final da fase 1)
            | x1      | x2      | x3      | x4      |
      0.000 |   0.000 |   0.000 |   0.000 |   0.000 |
-----------------------------------------------------
x1    1.000 |   1.000 |   0.000 |   0.000 |   0.500 |
x2    0.500 |   0.000 |   1.000 |   0.000 |  -0.750 |
x3    0.333 |   0.000 |   0.000 |   1.000 |   0.333 |

Simplex: Fase 2

Iteração 1
            | x1      | x2      | x3      | x4      |
     -1.833 |   0.000 |   0.000 |   0.000 |  -0.083 |
-----------------------------------------------------
x1    1.000 |   1.000 |   0.000 |   0.000 |   0.500 |
x2    0.500 |   0.000 |   1.000 |   0.000 |  -0.750 |
x3    0.333 |   0.000 |   0.000 |   1.000 |   0.333*|

Iteração 2
            | x1      | x2      | x3      | x4      |
     -1.750 |   0.000 |   0.000 |   0.250 |   0.000 |
-----------------------------------------------------
x1    0.500 |   1.000 |   0.000 |  -1.500 |   0.000 |
x2    1.250 |   0.000 |   1.000 |   2.250 |   0.000 |
x4    1.000 |   0.000 |   0.000 |   3.000 |   1.000 |

Solução ótima encontrada com custo 1.75000:
x =

   0.50000
   1.25000
   0.00000
   1.00000


---------------------------------------------------------------------
Exemplo 3 - Solução Ótima - Removendo Variáveis Artificiais na Fase 1
---------------------------------------------------------------------
m =  2
n =  2
A =

   1  -2
   2  -8

b =

   2
   4

c =

  -5
  -1

Simplex: Fase 1

Iteração 1
            | x1      | x2      | x3      | x4      |
     -6.000 |  -3.000 |  10.000 |   0.000 |   0.000 |
-----------------------------------------------------
x3    2.000 |   1.000*|  -2.000 |   1.000 |   0.000 |
x4    4.000 |   2.000 |  -8.000 |   0.000 |   1.000 |

Iteração 2
            | x1      | x2      | x3      | x4      |
      0.000 |   0.000 |   4.000 |   3.000 |   0.000 |
-----------------------------------------------------
x1    2.000 |   1.000 |  -2.000 |   1.000 |   0.000 |
x4    0.000 |   0.000 |  -4.000 |  -2.000 |   1.000 |

Iteração 3 (removendo variável artificial)
            | x1      | x2      |
      0.000 |   0.000 |   4.000 |
---------------------------------
x1    2.000 |   1.000 |  -2.000 |
x4    0.000 |   0.000 |  -4.000*|

Iteração 3 (tableau final da fase 1)
            | x1      | x2      |
      0.000 |   0.000 |   0.000 |
---------------------------------
x1    2.000 |   1.000 |   0.000 |
x2   -0.000 |  -0.000 |   1.000 |

Simplex: Fase 2

Iteração 1
            | x1      | x2      |
     10.000 |   0.000 |   0.000 |
---------------------------------
x1    2.000 |   1.000 |   0.000 |
x2   -0.000 |  -0.000 |   1.000 |

Solução ótima encontrada com custo -10.00000:
x =

   2
  -0


---------------------------------------
Exemplo 4 - Problema Inviável - Simples
---------------------------------------
m =  2
n =  2
A =

   1   2
   2   4

b =

   1
   3

c =

   1
   1

Simplex: Fase 1

Iteração 1
            | x1      | x2      | x3      | x4      |
     -4.000 |  -3.000 |  -6.000 |   0.000 |   0.000 |
-----------------------------------------------------
x3    1.000 |   1.000*|   2.000 |   1.000 |   0.000 |
x4    3.000 |   2.000 |   4.000 |   0.000 |   1.000 |

Iteração 2
            | x1      | x2      | x3      | x4      |
     -1.000 |   0.000 |   0.000 |   3.000 |   0.000 |
-----------------------------------------------------
x1    1.000 |   1.000 |   2.000 |   1.000 |   0.000 |
x4    1.000 |   0.000 |   0.000 |  -2.000 |   1.000 |

Problema inviável.

-----------------------------
Exemplo 5 - Problema Inviável
-----------------------------
m =  3
n =  5
A =

   -2   -3    5    3   -6
    1  -13    4    1   -7
    0    6    2    3    1

b =

   -9
  -12
    1

c =

  -1
  -2
   1
   2
   3

Simplex: Fase 1

Iteração 1
            | x1      | x2      | x3      | x4      | x5      | x6      | x7      | x8      |
    -22.000 |  -1.000 | -22.000 |   7.000 |   1.000 | -14.000 |   0.000 |   0.000 |   0.000 |
---------------------------------------------------------------------------------------------
x6    9.000 |   2.000*|   3.000 |  -5.000 |  -3.000 |   6.000 |   1.000 |   0.000 |   0.000 |
x7   12.000 |  -1.000 |  13.000 |  -4.000 |  -1.000 |   7.000 |   0.000 |   1.000 |   0.000 |
x8    1.000 |   0.000 |   6.000 |   2.000 |   3.000 |   1.000 |   0.000 |   0.000 |   1.000 |

Iteração 2
            | x1      | x2      | x3      | x4      | x5      | x6      | x7      | x8      |
    -17.500 |   0.000 | -20.500 |   4.500 |  -0.500 | -11.000 |   0.500 |   0.000 |   0.000 |
---------------------------------------------------------------------------------------------
x1    4.500 |   1.000 |   1.500 |  -2.500 |  -1.500 |   3.000 |   0.500 |   0.000 |   0.000 |
x7   16.500 |   0.000 |  14.500 |  -6.500 |  -2.500 |  10.000 |   0.500 |   1.000 |   0.000 |
x8    1.000 |   0.000 |   6.000*|   2.000 |   3.000 |   1.000 |   0.000 |   0.000 |   1.000 |

Iteração 3
            | x1      | x2      | x3      | x4      | x5      | x6      | x7      | x8      |
    -14.083 |   0.000 |   0.000 |  11.333 |   9.750 |  -7.583 |   0.500 |   0.000 |   3.417 |
---------------------------------------------------------------------------------------------
x1    4.250 |   1.000 |   0.000 |  -3.000 |  -2.250 |   2.750 |   0.500 |   0.000 |  -0.250 |
x7   14.083 |   0.000 |   0.000 | -11.333 |  -9.750 |   7.583 |   0.500 |   1.000 |  -2.417 |
x2    0.167 |   0.000 |   1.000 |   0.333 |   0.500 |   0.167*|   0.000 |   0.000 |   0.167 |

Iteração 4
            | x1      | x2      | x3      | x4      | x5      | x6      | x7      | x8      |
     -6.500 |   0.000 |  45.500 |  26.500 |  32.500 |   0.000 |   0.500 |   0.000 |  11.000 |
---------------------------------------------------------------------------------------------
x1    1.500 |   1.000 | -16.500 |  -8.500 | -10.500 |   0.000 |   0.500 |   0.000 |  -3.000 |
x7    6.500 |   0.000 | -45.500 | -26.500 | -32.500 |   0.000 |   0.500 |   1.000 | -10.000 |
x5    1.000 |   0.000 |   6.000 |   2.000 |   3.000 |   1.000 |   0.000 |   0.000 |   1.000 |

Problema inviável.

----------------------------------------
Exemplo 6 - Problema Ilimitado - Simples
----------------------------------------
m =  1
n =  3
A =

   0   1   1

b =  1
c =

  -1
   0
   0

Simplex: Fase 1

Iteração 1
            | x1      | x2      | x3      | x4      |
     -1.000 |  -0.000 |  -1.000 |  -1.000 |   0.000 |
-----------------------------------------------------
x4    1.000 |   0.000 |   1.000*|   1.000 |   1.000 |

Iteração 2
            | x1      | x2      | x3      | x4      |
      0.000 |   0.000 |   0.000 |   0.000 |   1.000 |
-----------------------------------------------------
x2    1.000 |   0.000 |   1.000 |   1.000 |   1.000 |

Iteração 3 (tableau final da fase 1)
            | x1      | x2      | x3      |
      0.000 |   0.000 |   0.000 |   0.000 |
-------------------------------------------
x2    1.000 |   0.000 |   1.000 |   1.000 |

Simplex: Fase 2

Iteração 1
            | x1      | x2      | x3      |
     -0.000 |  -1.000 |   0.000 |   0.000 |
-------------------------------------------
x2    1.000 |   0.000 |   1.000 |   1.000 |

Problema ilimitado.

------------------------------
Exemplo 7 - Problema Ilimitado
------------------------------
m =  3
n =  5
A =

  -5   1   1   0   0
  -2   1   0  -1   0
  -1   1   0   0  -1

b =

   1
  -1
  -2

c =

   1
  -1
   1
   1
  -1

Simplex: Fase 1

Iteração 1
            | x1      | x2      | x3      | x4      | x5      | x6      | x7      | x8      |
     -4.000 |   2.000 |   1.000 |  -1.000 |  -1.000 |  -1.000 |   0.000 |   0.000 |   0.000 |
---------------------------------------------------------------------------------------------
x6    1.000 |  -5.000 |   1.000 |   1.000*|   0.000 |   0.000 |   1.000 |   0.000 |   0.000 |
x7    1.000 |   2.000 |  -1.000 |  -0.000 |   1.000 |  -0.000 |   0.000 |   1.000 |   0.000 |
x8    2.000 |   1.000 |  -1.000 |  -0.000 |  -0.000 |   1.000 |   0.000 |   0.000 |   1.000 |

Iteração 2
            | x1      | x2      | x3      | x4      | x5      | x6      | x7      | x8      |
     -3.000 |  -3.000 |   2.000 |   0.000 |  -1.000 |  -1.000 |   1.000 |   0.000 |   0.000 |
---------------------------------------------------------------------------------------------
x3    1.000 |  -5.000 |   1.000 |   1.000 |   0.000 |   0.000 |   1.000 |   0.000 |   0.000 |
x7    1.000 |   2.000*|  -1.000 |   0.000 |   1.000 |   0.000 |   0.000 |   1.000 |   0.000 |
x8    2.000 |   1.000 |  -1.000 |   0.000 |   0.000 |   1.000 |   0.000 |   0.000 |   1.000 |

Iteração 3
            | x1      | x2      | x3      | x4      | x5      | x6      | x7      | x8      |
     -1.500 |   0.000 |   0.500 |   0.000 |   0.500 |  -1.000 |   1.000 |   1.500 |   0.000 |
---------------------------------------------------------------------------------------------
x3    3.500 |   0.000 |  -1.500 |   1.000 |   2.500 |   0.000 |   1.000 |   2.500 |   0.000 |
x1    0.500 |   1.000 |  -0.500 |   0.000 |   0.500 |   0.000 |   0.000 |   0.500 |   0.000 |
x8    1.500 |   0.000 |  -0.500 |   0.000 |  -0.500 |   1.000*|   0.000 |  -0.500 |   1.000 |

Iteração 4
            | x1      | x2      | x3      | x4      | x5      | x6      | x7      | x8      |
      0.000 |   0.000 |   0.000 |   0.000 |   0.000 |   0.000 |   1.000 |   1.000 |   1.000 |
---------------------------------------------------------------------------------------------
x3    3.500 |   0.000 |  -1.500 |   1.000 |   2.500 |   0.000 |   1.000 |   2.500 |   0.000 |
x1    0.500 |   1.000 |  -0.500 |   0.000 |   0.500 |   0.000 |   0.000 |   0.500 |   0.000 |
x5    1.500 |   0.000 |  -0.500 |   0.000 |  -0.500 |   1.000 |   0.000 |  -0.500 |   1.000 |

Iteração 5 (tableau final da fase 1)
            | x1      | x2      | x3      | x4      | x5      |
      0.000 |   0.000 |   0.000 |   0.000 |   0.000 |   0.000 |
---------------------------------------------------------------
x3    3.500 |   0.000 |  -1.500 |   1.000 |   2.500 |   0.000 |
x1    0.500 |   1.000 |  -0.500 |   0.000 |   0.500 |   0.000 |
x5    1.500 |   0.000 |  -0.500 |   0.000 |  -0.500 |   1.000 |

Simplex: Fase 2

Iteração 1
            | x1      | x2      | x3      | x4      | x5      |
     -2.500 |   0.000 |   0.500 |   0.000 |  -2.500 |   0.000 |
---------------------------------------------------------------
x3    3.500 |   0.000 |  -1.500 |   1.000 |   2.500 |   0.000 |
x1    0.500 |   1.000 |  -0.500 |   0.000 |   0.500*|   0.000 |
x5    1.500 |   0.000 |  -0.500 |   0.000 |  -0.500 |   1.000 |

Iteração 2
            | x1      | x2      | x3      | x4      | x5      |
      0.000 |   5.000 |  -2.000 |   0.000 |   0.000 |   0.000 |
---------------------------------------------------------------
x3    1.000 |  -5.000 |   1.000*|   1.000 |   0.000 |   0.000 |
x4    1.000 |   2.000 |  -1.000 |   0.000 |   1.000 |   0.000 |
x5    2.000 |   1.000 |  -1.000 |   0.000 |   0.000 |   1.000 |

Iteração 3
            | x1      | x2      | x3      | x4      | x5      |
      2.000 |  -5.000 |   0.000 |   2.000 |   0.000 |   0.000 |
---------------------------------------------------------------
x2    1.000 |  -5.000 |   1.000 |   1.000 |   0.000 |   0.000 |
x4    2.000 |  -3.000 |   0.000 |   1.000 |   1.000 |   0.000 |
x5    3.000 |  -4.000 |   0.000 |   1.000 |   0.000 |   1.000 |

Problema ilimitado.
\end{verbatim}
\end{tiny}

%%%%%%%%%%%%%%%%%%%%%%%%%%%%%%%%%%%%%%%%%%%%%%%%%%%%%%%%%%%%%%%%%%%%%%%%%%%%%%%%%%%%%%%%%%%%%%%%%%%%

\begin{thebibliography}{99}
	\bibitem{bertsimas} D.~Bertsimas \& J.~N.~Tsitsiklis, {\it Introduction to Linear Optimization}, 
	1997, Athena Scientific.
\end{thebibliography}

%%%%%%%%%%%%%%%%%%%%%%%%%%%%%%%%%%%%%%%%%%%%%%%%%%%%%%%%%%%%%%%%%%%%%%%%%%%%%%%%%%%%%%%%%%%%%%%%%%%%

\end{document}

