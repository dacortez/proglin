% Tipo de documento
\documentclass[12pt,a4paper]{article}

% Pacotes
%\usepackage{showlabels}
\usepackage{latexsym}
\usepackage{amsfonts}
\usepackage{amsmath}
\usepackage{amscd}
\usepackage[brazil]{babel}
\usepackage[utf8]{inputenc}
\usepackage[pdftex]{graphicx}

% Paginação
%\textwidth 15.0cm                    % Largura
%\textheight 22.0cm                   % Altura
%\addtolength{\oddsidemargin}{0.0cm}  % Margem esquerda (impar)
%\addtolength{\evensidemargin}{0.0cm} % Margem esquerda (par)
%\addtolength{\topmargin}{0.0cm}      % Margem superior

% Estilo dos parágrafos
\sloppy                              % Mais flexível
\setlength{\jot}{13pt}               % Distância entre linhas do eqnarray
\setlength{\parskip}{1ex}            % Distância entre parágrafos
\renewcommand{\baselinestretch}{1.0} % Distância entre linhas

% Comandos customizados
\newcommand{\vet}{\mathbf}                                   % vetor
\newcommand{\ie}{\emph{i.e.}}                                % isto é
\newcommand{\prodint}[2]{\left\langle #1 , #2 \right\rangle} % produto interno
\newcommand{\Fullbox}{{\rule{2.0mm}{2.0mm}}}                 % caixa cheia
\newcommand{\EOS}{\hfill\Box\vspace{-0.2cm}}                 % fim de enunciado
\newcommand{\EOP}{\hfill\Fullbox\vspace{0.2cm}}              % fim de prova
\newcommand{\eps}{\varepsilon}                               % epsilon
\newcommand{\defi}{\: := \: }                                % definição
\newcommand{\del}{\partial}                                  % derivada parcial

% Contagem de equações por seção
\renewcommand{\theequation}{\thesection.\arabic{equation}}

% Contagem de figuras por seção
\renewcommand{\thefigure}{\thesection.\arabic{figure}}

% Contagem de tabelas por seção
\renewcommand{\thetable}{\thesection.\arabic{table}}

% Zerar as contagem em cada seção
\newcommand{\zerar}{\setcounter{equation}{0}\setcounter{figure}{0}\setcounter{table}{0}}

% Operadores
\DeclareMathOperator{\sen}{sen}
\DeclareMathOperator{\tg}{tg}
\DeclareMathOperator{\cotg}{cotg}
\DeclareMathOperator{\im}{im}
\DeclareMathOperator{\arctg}{arctg}
\DeclareMathOperator{\diag}{diag}
\DeclareMathOperator{\sgn}{sgn}
\DeclareMathOperator{\tr}{tr}

% Conjunto de números
\newcommand{\Z}{\mathbb{Z}}
\newcommand{\C}{\mathbb{C}}
\newcommand{\N}{\mathbb{N}}
\newcommand{\Q}{\mathbb{Q}}
\newcommand{\R}{\mathbb{R}}

%%%%%%%%%%%%%%%%%%%%%%%%%%%%%%%%%%%%%%%%%%%%%%%%%%%%%%%%%%%%%%%%%%%%%%%%%%%%%%%%%%%%%%%%%%%%%%%%%%%%

\begin{document}

\title{Programação Linear \\ Relatório EP3 -- Método Simplex}
\author{Daniel Augusto Cortez -- 2960291 \\ Lucas Rodrigues Colucci -- 6920251}
\date{\today}

\maketitle

%%%%%%%%%%%%%%%%%%%%%%%%%%%%%%%%%%%%%%%%%%%%%%%%%%%%%%%%%%%%%%%%%%%%%%%%%%%%%%%%%%%%%%%%%%%%%%%%%%%%

\zerar
\section{Introdução}
\label{sec:introducao}

Apresentamos neste relatório a implementação do método simplex (full tableau), utilizando a 
linguagem de programação Octave. O algoritmo é desenvolvido em detalhes em~\cite{bertsimas}.
Considera-se o problema de programação linear em seu formato padrão
%
\begin{equation*}
	\begin{array}{rl}
		\text{minimizar} & \vet{c}^T \vet{x} \\
		\text{sujeito a} & \vet{A} \vet{x} = \vet{b} \\\
		                 & \vet{x} \geq \vet{0} \, ,
	 \end{array}
\end{equation*}
%
onde $\vet{x}, \vet{c} \in \R^n$, $\vet{b} \in \R^m$ e $\vet{A} \in \R^{m \times n}$. 
O algoritmo pode ser descrito pelos seguintes passos:

\vspace{0.5cm}

\noindent {\bf Fase I.}
\vspace{-0.2cm}
\begin{enumerate}
	\item Multiplique algumas das restrições por $-1$, modificando o problema de tal forma que  
	$\vet{b} \geq 0$.
	\item Introduza variáveis artificiais $y_1, \ldots, y_m$ e aplique o método simplex ao problema
	auxiliar com custo $\sum_{i=1}^{m}y_i$.
	\item Se o custo ótimo do problema auxiliar é positivo, o problema original é inviável e o 
	algoritmo termina.
	\item Se o custo ótimo do problema auxiliar é nulo, uma solução viável do problema original é 
	obtida. Se nenhuma variável artificial está na base final, as variáveis artificiais e respectivas
	colunas são eliminadas, e uma base viável para o problema original está disponível.
	\item Se a $\ell$-ésima variável básica é artificial, examine a $\ell$-ésima entrada das
	colunas $\vet{B}^{-1}\vet{A}_j$, $j = 1, \ldots, n$. Se todas as entradas são nulas, então a 
	$\ell$-ésima linha representa uma restrição redundante e é eliminada. De outra forma, se a 
	$\ell$-ésima entrada da $j$-ésima coluna é diferente de zero, aplique uma mudança de base:
	a $\ell$-ésima variável básica sai e $x_j$ entra na base. Repita esse procedimento até que todas 
	as variáveis artificiais sejam levadas para fora da base.
\end{enumerate}

\noindent {\bf Fase II.}
\vspace{-0.2cm}
\begin{enumerate}
	\item Faça a base e o tableau final obtidos na fase I serem a base e o tableau inicial para a 
	fase~II.
	\item Calcule o custo reduzido de todas as variáveis para essa base inicial, usando o vetor de 
	custos do problema original.
	\item Aplique o método simplex ao problema original.
\end{enumerate}

\noindent {\bf Uma iteração da implementação através do tableau.}
\vspace{-0.2cm}
\begin{enumerate}
	\item Uma iteração típica começa com um tableau associado com uma matriz básica $\vet{B}$ e a 
	solução básica correspondente $\vet{x}$.
	\item Examine todos os custos reduzidos na linha zero do tableau. Se todos forem não-negativos,
	a solução viável básica atual é ótima, e o algoritmo termina. Caso contrário, escolha algum $j$
	tal que $\overline{c}_j < 0$.
	\item Considere o vetor $\vet{u} = \vet{B}^{-1}\vet{A}_j$, que é a $j$-ésima coluna (a coluna 
	pivô) do tableau. Se nenhuma componente de $\vet{u}$ é positiva, o custo ótimo é $-\infty$ e o 
	algoritmo termina.
	\item Para cada $i$ tal que $u_i > 0$, calcule a razão $x_{B(i)}/u_i$. Seja $\ell$ o índice da 
	linha que corresponde à menor razão. A coluna $\vet{A}_{B(\ell)}$ saí da base e a coluna 
	$\vet{A}_j$ entra.
	\item Adicione a cada linha do tableau um múltiplo constante da $\ell$-ésima linha tal que o 
	$u_\ell$ (elemento pivô) se torne um e todas as outras entradas da coluna pivô se tornem zero. 
\end{enumerate}

A implementação do método simplex em Octave seguiu extamente os passos acima, conforme descrevemos
a seguir.

%%%%%%%%%%%%%%%%%%%%%%%%%%%%%%%%%%%%%%%%%%%%%%%%%%%%%%%%%%%%%%%%%%%%%%%%%%%%%%%%%%%%%%%%%%%%%%%%%%%%

\zerar
\section{Funções Implementadas}
\label{sec:funcoes}

%%%%%%%%%%%%%%%%%%%%%%%%%%%%%%%%%%%%%%%%%%%%%%%%%%%%%%%%%%%%%%%%%%%%%%%%%%%%%%%%%%%%%%%%%%%%%%%%%%%%

\zerar
\section{Alguns Exemplos}
\label{sec:exemplos}

%%%%%%%%%%%%%%%%%%%%%%%%%%%%%%%%%%%%%%%%%%%%%%%%%%%%%%%%%%%%%%%%%%%%%%%%%%%%%%%%%%%%%%%%%%%%%%%%%%%%

\zerar
\section{Utilização do Programa}
\label{sec:programa}

%%%%%%%%%%%%%%%%%%%%%%%%%%%%%%%%%%%%%%%%%%%%%%%%%%%%%%%%%%%%%%%%%%%%%%%%%%%%%%%%%%%%%%%%%%%%%%%%%%%%

\begin{thebibliography}{99}
	\bibitem{bertsimas} D.~Bertsimas \& J.~N.~Tsitsiklis, {\it Introduction to Linear Optimization}, 
	1997, Athena Scientific.
\end{thebibliography}

%%%%%%%%%%%%%%%%%%%%%%%%%%%%%%%%%%%%%%%%%%%%%%%%%%%%%%%%%%%%%%%%%%%%%%%%%%%%%%%%%%%%%%%%%%%%%%%%%%%%

\end{document}

